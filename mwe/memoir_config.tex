
%%%%%%%%%%
% VARIA
%%%%%%%%%%
\usepackage{mwe}
\usepackage{adjustbox}
\usepackage{pdfpages}\usepackage{ifoddpage}
\usepackage{enumitem}

\newcommand{\R}{\textsuperscript{\textregistered}}
\newcommand{\HRule}[1]{{\color{Blue1}\rule{\textwidth}{#1}}}

\newcommand{\pr}{$^\prime$}
\newcommand*{\mydprime}{^{\prime\prime}\mkern-1.2mu}
\newcommand{\dpr}{$\mydprime$}
%%%%%%%%%%
% PAGINA SETUP
%%%%%%%%%%
\usepackage[utf8]{inputenc}

\usepackage{xcolor}
% KLEINERE MARGES
\definecolor{BlueLogo1}{RGB}{82,189,236}
\definecolor{BlueLogo2}{RGB}{0,64,122}
\definecolor{Blue1}{RGB}{31,171,231}
\definecolor{Blue2}{RGB}{29,141,176}
\definecolor{Blue3}{RGB}{17,110,138}
\colorlet{niceGray}{black!20}

\usepackage[
marginratio={4:6,5:7},
textwidth=157.5mm, textheight=234mm,
headheight=\baselineskip,
footskip=1.2cm,
bindingoffset=6mm
]{geometry}

\usepackage{lscape}

%%%%%%%%%%%%%%
% PAGINA LAY-OUT
%%%%%%%%%%%%%%

% the chapter style:
\usepackage{xcoffins,xcolor}
\NewCoffin\main
\NewCoffin\titleline
\NewCoffin\chapternumber

\makechapterstyle{Bringhurst}{%
  \renewcommand*\chapterheadstart{}
  \renewcommand*\printchaptername{}
  \renewcommand*\chapternamenum{}
  \renewcommand*\afterchapternum{}
  % numbered chapters:
  \renewcommand*\printchapternum{%
      \SetHorizontalCoffin\chapternumber{%
     \sffamily\textcolor{Blue1}{\thechapter}%
    }%
    \ScaleCoffin\chapternumber{6}{6}%
  }
  % unnumbered chapters:
  \renewcommand*\printchapternonum{\SetHorizontalCoffin\chapternumber{}}
  \renewcommand*\printchaptertitle[1]{
  	\SetVerticalCoffin\main{\textwidth}{
  	\setSpacing{2}\memRTLraggedright{\color{Blue1}\huge\sffamily\MakeUppercase{\textls[75]{##1}}}
  	}
  }
  \renewcommand*\afterchaptertitle{%
    \vskip\onelineskip 
    \SetHorizontalCoffin\titleline{}
    \JoinCoffins\main[r,t]\chapternumber[l,t](\marginparsep,0pt)%
    \TypesetCoffin\main
    \vskip\onelineskip
  }
}


\chapterstyle{Bringhurst}

% sections and subsections:
\setsecnumformat{\normalfont\sffamily\large \csname the#1\endcsname  \quad}

% the section style:
\newcommand\uppercasehead[1]{%
  \noindent{\large\sffamily\scshape{\textls[50]{#1}}}}
\setsecindent{0pt}
\setsecheadstyle{\uppercasehead}

% the subsection style:
\newcommand\itshapehead[1]{\large\sffamily\itshape#1}
\setsubsecheadstyle{\sffamily}
\setsecnumdepth{subsection}

% the subsubsection style:
\setsubsubsecheadstyle{\color{Blue1}$\triangleright$\quad\sffamily}%\itshapehead

%headers
\makepagestyle{myruledpagestyle}
\makeevenhead{myruledpagestyle}{\textcolor{black}{\leftmark}}{}{}%\thepage x x
\makeoddhead{myruledpagestyle}{}{}{\textcolor{black}{\scshape\rightmark}}%x x \thepage
\makeevenfoot{myruledpagestyle}{\thepage}{}{}%\theversion}
\makeoddfoot{myruledpagestyle}{}{}{\thepage}%{\theversion
\makeatletter
\makepsmarks{myruledpagestyle}{
  \def\chaptermark##1{\markboth{%
        \ifnum \value{secnumdepth} > -1
          \if@mainmatter
           \chaptername\ \thechapter\quad%
          \fi
        \fi
        ##1}{}}
  \def\sectionmark##1{\markright{%
        \ifnum \value{secnumdepth} > 0
          \thesection \ \;%
        \fi
        ##1}}
}
\makeatother

\newlength{\mylength}
\setlength{\mylength}{\textwidth}%34.4pc (40) (42.4pc)
\makerunningwidth{myruledpagestyle}{\mylength}
\makeheadposition{myruledpagestyle}{}{}{}{}%{flushright}{flushleft}{flushright}{flushleft}
\makeheadrule{myruledpagestyle}{\mylength}{0.25pt}
\makeheadfootruleprefix{myruledpagestyle}{\color{Blue1}}

\makepagestyle{mypagestyle}
\makeoddfoot{mypagestyle}{}{}{\thepage}
\makerunningwidth{mypagestyle}{\textwidth}%34.4pc
\makeheadposition{myruledpagestyle}{flushright}{flushleft}{flushright}{flushleft}
\aliaspagestyle{chapter}{mypagestyle}

%headers - front/backmatter
\makepagestyle{mynotruledpagestyle}
%\makeevenhead{myruledpagestyle}{\textcolor{black}{\leftmark}}{}{}%\thepage x x
%\makeoddhead{myruledpagestyle}{}{}{\textcolor{black}{\scshape\rightmark}}%x x \thepage
\makeevenfoot{mynotruledpagestyle}{\thepage}{}{}%\theversion}
\makeoddfoot{mynotruledpagestyle}{}{}{\thepage}%{\theversion
%\makeatletter
%\makepsmarks{myruledpagestyle}{
%  \def\chaptermark##1{\markboth{%
%        \ifnum \value{secnumdepth} > -1
%          \if@mainmatter
%           \chaptername\ \thechapter\quad%
%          \fi
%        \fi
%        ##1}{}}
%  \def\sectionmark##1{\markright{%
%        \ifnum \value{secnumdepth} > 0
%          \thesection \ \;%
%        \fi
%        ##1}}
%}
%\makeatother

%%%%%%%%%%%%%%
% LETTERTYPE
%%%%%%%%%%%%%%

\usepackage[dutch,main=english]{babel}
\usepackage{lipsum}

%\usepackage{libertine}

\usepackage[no-math]{fontspec}
\setmainfont[WordSpace=1.5]{Erewhon}
\setsansfont{Source Sans Pro}
\setmonofont{Source Code Pro}

% we want to letterspace uppercased words and those in small caps, so:
\usepackage{microtype}

% spacing
\OnehalfSpacing

%%%%%%%%%%%%%%
% MARGE INHOUD
%%%%%%%%%%%%%%

\usepackage{marginfix}

%%%%%%%%%%%%%%
% TOC
%%%%%%%%%%%%%%
\settocdepth{subsection}

% aanpassen TOC aan nieuwe chapter/section heading
\renewcommand{\cftchapterfont}{\color{Blue1}\sffamily\scshape}
\renewcommand{\cftsectionfont}{\sffamily\scshape}  
\renewcommand{\cftsubsectionfont}{\sffamily}  

% aanpassen paginanummers
\renewcommand{\cftchapterpagefont}{\normalfont\sffamily}
\renewcommand{\cftsectionpagefont}{\normalfont\sffamily}
\renewcommand{\cftsubsectionpagefont}{\normalfont\sffamily}

%aanpassen nummers
\renewcommand{\cftchapterpresnum}{\color{Blue1}\normalfont\sffamily}
\renewcommand{\cftsectionpresnum}{\normalfont\sffamily}
\renewcommand{\cftsubsectionpresnum}{\normalfont\sffamily}

%aanpassen ruimte boven chapters
\renewcommand{\cftbeforechapterskip}{10pt}

%%%%%%%%%%%%%%%
% BIBLIOGRAFIE
%%%%%%%%%%%%%%%

\usepackage[style=chem-angew,sorting=none,url=false,doi=false,isbn=false,eprint=false, maxbibnames=999]{biblatex}%nature, backend=biber
 
 %%%%%%%%%%%%%%%
 % CHEMISTRY
 %%%%%%%%%%%%%%%

\usepackage[version=3]{mhchem}    

%%%%%%%%%%%%%%%%%
% GRAPHICS/TABLES
%%%%%%%%%%%%%%%%%
\usepackage{graphicx}
\usepackage{threeparttable}
\usepackage{nicematrix}
\usepackage{longtable}
\usepackage{colortbl}

\usepackage{dcolumn}
\makeatletter
\newcolumntype{B}[3]{>{\boldmath\DC@{#1}{#2}{#3}}c<{\DC@end}}
\makeatother

\usepackage{wrapfig}
\usepackage{booktabs}

    \newcommand{\midsepremove}{\aboverulesep = 0mm \belowrulesep = 0mm}
    \midsepremove
    \newcommand{\midsepdefault}{\aboverulesep = 0.605mm \belowrulesep = 0.984mm}
    \midsepdefault

\usepackage{multicol}
\usepackage{multirow}

% GRAPHICS PATH
\graphicspath{
{./figures/}
}


\usepackage[skip=0pt]{caption}     
\DeclareCaptionJustification{single}{\SingleSpacing}
\captionsetup[figure]{format=plain,textfont={footnotesize},justification=single,labelfont={color={Blue1},sc,sf,small},labelsep=space,singlelinecheck=off}
\captionsetup[table]{format=plain,font={footnotesize},labelfont={color={Blue1},sc,sf,small},justification=single,labelsep=space}
\captionsetup[lstlisting]{format=plain,font={footnotesize},justification=single,labelfont={color={Blue1},sc,sf,small},labelsep=space,singlelinecheck=off}

\usepackage{xparse}

%%%%%%%
% TYPESET CODE
%%%%%%%
\usepackage{listings}

\definecolor{dkgreen}{rgb}{0,0.6,0}
\definecolor{gray}{rgb}{0.5,0.5,0.5}
\definecolor{mauve}{rgb}{0.58,0,0.82}
\definecolor{backcolour}{rgb}{0.95,0.95,0.92}

\lstloadlanguages{Awk,Matlab,sh}

\lstdefinestyle{plain}{
 commentstyle=\color{dkgreen},
 keywordstyle=\color{blue},
 numberstyle=\tiny\color{gray},
 stringstyle=\color{mauve},
 basicstyle={\footnotesize\ttfamily}, 
 breakatwhitespace=true,
 breaklines=true,
 captionpos=top,
 keepspaces=true,
 numbers=left,
 numbersep=5pt,
 showspaces=false,
 showstringspaces=false,
 showtabs=false,
 tabsize=1,
}

\lstset{style=plain}

\lstdefinelanguage{amber_inp}
{
  % list of keywords
  morekeywords={
  pmemd.MPI
  pmemd
  sander
  sander.MPI
  pmemd.cuda
  },
  sensitive=true, % keywords are not case-sensitive
  morecomment=[l]{!}, % l is for line comment
  morecomment=[s]{/*}{*/}, % s is for start and end delimiter
  morestring=[b]" % defines that strings are enclosed in double quotes
}

%%%%
% BLOCKQUOTE
%%%%
\usepackage{etoolbox}
\usepackage{framed}
\usepackage{tikz}

% Make commands for the quotes
\newcommand*{\openquote}
   {\tikz[remember picture,overlay,xshift=-4ex,yshift=-2.5ex]
   \node (OQ) {{\color{white}\fontsize{45}{45}\selectfont``}};\kern0pt}

\newcommand*{\closequote}[1]
  {\tikz[remember picture,overlay,xshift=4ex,yshift={#1}]
   \node (CQ) {{\color{white}\fontsize{45}{45}\selectfont''}};}
   
% select a colour for the shading
\colorlet{shadecolor}{Blue1!35}

\newcommand*\shadedauthorformat{\emph} % define format for the author argument

% Now a command to allow left, right and centre alignment of the author
\newcommand*\authoralign[1]{%
  \if#1l
    \def\authorfill{}\def\quotefill{\hfill}
  \else
    \if#1r
      \def\authorfill{\hfill}\def\quotefill{}
    \else
      \if#1c
        \gdef\authorfill{\hfill}\def\quotefill{\hfill}
      \else\typeout{Invalid option}
      \fi
    \fi
  \fi}

\usepackage{tcolorbox}
  
  % wrap everything in its own environment which takes one argument (author) and one optional argument
% specifying the alignment [l, r or c]
%
\newenvironment{shadequote}[2][l]%
{\authoralign{#1}
\ifblank{#2}
   {\def\shadequoteauthor{}\def\yshift{-2ex}\def\quotefill{\hfill}}
   {\def\shadequoteauthor{\par\authorfill\shadedauthorformat{#2}}\def\yshift{2ex}}
\begin{snugshade}\begin{quote}\openquote}
{\shadequoteauthor\quotefill\closequote{\yshift}\end{quote}\end{snugshade}}


%%%%%%%%%%%%%%%%%
% UNITS
%%%%%%%%%%%%%%%%%

\usepackage[binary-units, per-mode=symbol, version-1-compatibility, group-minimum-digits=4,detect-all]{siunitx} 
\sisetup{range-units=single,separate-uncertainty=true,multi-part-units=single}
\DeclareSIUnit[number-unit-product = {\,}]\cal{cal}
\DeclareSIUnit{\calorie}{cal}

%%%%%%%%%%%%%%%%%
% Maths
%%%%%%%%%%%%%%%%%

\usepackage{physics}
\usepackage{unicode-math}
\setmathfont{Erewhon Math}
\usepackage{nicefrac}

%fix bold math in list of acronyms 
%TO DO: make this work
\makeatletter
\g@addto@macro\bfseries{\boldmath}
\makeatother

%%%%%%
% HYPERREFERENCING
%%%%%%
%\renewcommand\AtEndPackage[2]{\def\temp{\AddToHook{package/after/#1}{#2}}\temp}
\usepackage[hyperfootnotes=false,hidelinks]{hyperref}
%\usepackage{memhfixc}

%%%%%%%%%%%%%%%%%
% CROSS-REFERENCING
%%%%%%%%%%%%%%%%%

\usepackage[dutch,english]{cleveref} %\cref; \Cref -> capital; \cref{1,2} -> multiple references %[noabbrev]
\AtBeginDocument{
	\crefname{subsubsection, subsection, section}{\S}{\S}
	\crefname{equation}{Equation}{Equations}
	\crefname{chapter}{Chapter}{Chapters}
	\crefname{table}{Table}{Tables}
	\crefname{figure}{Figure}{Figures}
	\crefname{section}{\S}{\S}
	\crefname{subsection}{\S}{\S}
}	

\creflabelformat{equation}{#2#1#3}

%%%%%%%%%%
% GLOSSARIES
%%%%%%%%%%
\usepackage[acronym,toc,nonumberlist,order=letter,nogroupskip,nopostdot,nonumberlist]{glossaries}
%\usepackage{glossary-superragged}


