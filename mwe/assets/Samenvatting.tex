\chapter*{Samenvatting}

%In \foreignlanguage{dutch}{\cref{intro:chap}} wordt een introductie gegeven over nucleïnezuren en de gebruikte computationele methoden in dit werk. De biologische functie van nucleïnezuren wordt kort aangehaald waarna de link gelegd wordt met hun structuur. Bijkomstig wordt ook aandacht geschonken aan de verschillende mogelijkheden om wijzigingen in de chemische structuur van nucleïnezuren aan te brengen. Computationele methoden om chemische structuren te bestuderen kunnen in drie categorieën opgedeeld worden op basis van de nauwkeurigheid en detail die deze methoden bieden. De basis voor de methoden die structuren op atomair niveau analyseren worden bondig uitgelegd opdat de lezer een beter begrip van deze methoden zou verkrijgen.
%
%\foreignlanguage{dutch}{\Cref{aims:chap}} legt de beweegredenen voor dit doctoraatsproject uit die vertaald werden in specifieke doelen.
%
%In het eerste experimentele deel, \foreignlanguage{dutch}{\cref{puckering:chap}}, wordt een methode voorgesteld voor de conformationele analyse van de suikerring in (gemodificeerde) nucleosiden. De methode is gebaseerd op gekende computationele technieken om hiervan zeer nauwkeurige potenti\"ele energie oppervlakken van te maken. De efficiëntie van de voorgestelde methode is een bijkomstig voordeel waardoor ze ook gebruikt kan worden indien er geen toegang tot uitgebreide server systemen mogelijk is. De methode kan alsdusdanig gebruikt worden bij het valideren van krachtvelden in moleculaire mechanica. Het belang van zo'n validatie wordt duidelijk gemaakt in \foreignlanguage{dutch}{\cref{hna:chap}}.
%
%\foreignlanguage{dutch}{\Cref{hna:chap}} kan opgedeeld worden in twee delen. In het eerste deel wordt het krachtveld in de populaire moleculaire dynamica (MD) software \gls{amber} gevalideerd voor \gls{hna} moleculen. De analyse toont aan dat een simpele extrapolatie van het huidig krachtveld verkeerde resultaten met betrekking tot de conformaties van de suikerring oplevert vergeleken met de huidige experimentele kennis omtrent \gls{hna} moleculen. Een eerste poging tot herparameterisatie van het krachtveld was ondernomen om dit te corrigeren. Deze verfijning van het krachtveld werd in het tweede deel gebruikt om te helpen in het genereren van een moleculair model van een \gls{hna} aptameer. Hiertoe werden specifieke technieken in de moleculaire dynamica gebruikt die uitgingen van bepaalde experimentele resultaten. De stabiliteit van het uiteindelijke model werd verder verfijnd door het te simuleren gedurende een langere periode (\SI{1}{\micro\second}) in expliciet solvent.
%
%De karakterisatie van ``haarspeld'' structuren gevormd door (deoxy)xylose nucle\"inezuren vormt de inhoud van \foreignlanguage{dutch}{\cref{xylo:chap}} van deze thesis. Chemisch gesynthetiseerde oligomeren met verschillende lengtes (12, 14 en 20 residuen) werden geanalyseerd door zowel circulair dichro\"isme als thermische analyse de beiden aangaven dat deze sequenties aanleiding gaven tot het vormen van zogenaamde ``haarspeld'' structuren. De moleculaire structuur van een oligomeer van 20 residuen, die volledig bestond uit deoxyxylose residuen of uit een hybride variant, werd berekend op basis van nucleaire magnetische resonantie (NMR) en MD technieken. Deze structuren toonden aan dat de verhoogde thermische stabiliteit niet veroorzaakt zou kunnen worden door bijkomstige interne interacties in het duplex-deel van de sequentie. Deze analyse toonde ook aan dat het vormen van de ``haarspeld'' verhindert dat de sequentie verschillende structuren zou aannemen.
%
%In het laatste hoofdstuk, \foreignlanguage{dutch}{\cref{perspectives:chap}}, wordt een bondige algemene discussie over het voorgestelde werk geleverd samen met een vooruitzicht naar eventuele uitbreidingen van dit werk in de toekomst.
%%The final chapter, \cref{perspectives:chap}, contains a general discussion of the work presented in the dissertation together with some final remarks and a perspective for the future.

Here you put the summary in Dutch. Since \verb=\cref= or \verb=\Cref= (at start of a sentence) is used to crossreference tables/figures/chapters in the manuscript, you can add\newline \verb=\foreignlanguage{dutch}{\cref{...}}= to automatically translate the crossreference label, e.g. \foreignlanguage{dutch}{\cref{intro:chap}}.

Abrreviations are not used or manually handled, i.e. not through \verb=\gls=.