\chapter*{Summary}

This is the place to summarize your dissertation. Abbreviations, e.g. \gls{md} are added throughout the document and thus also in the summary. \verb=\glsresetall= is added to reset the first use of abbreviations for the actual body.

\glsresetall